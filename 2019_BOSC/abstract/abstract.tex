\documentclass[10pt,oneside]{article}

%%%%%%%%%%%%%
\setlength{\textheight}{8.75in} %Letter is 11in, less 2 for margins, less 0.25 for footer
\setlength{\oddsidemargin}{0.0in} %gets +1inc
\setlength{\evensidemargin}{0.0in} %gets +1inch
\setlength{\textwidth}{6.50in} %Letter is 8.5, less 2 inches for margins
\setlength{\topmargin}{0.5in}
\setlength{\headheight}{0in}
\setlength{\headsep}{0in}
\setlength{\parindent}{0.25in}
%%%%%%%%%%%%

% use letters instead of symbols to accommodate >7 authors
\makeatletter
\let\@fnsymbol\@alph
\makeatother

\usepackage[utf8]{inputenc}
\usepackage[numbers]{natbib}
\usepackage{graphicx}
\usepackage[colorlinks=true,citecolor=black,urlcolor=blue]{hyperref}

\title{%Hack to get the logo on the PDF front page:
\vspace{-1.5in}
\includegraphics[width=0.3\textwidth]{biopython_logo_s.png} \\
\vspace{3mm}Biopython Project Update 2019}
\author{
    Peter Cock\thanks{Information and Computational Sciences, James Hutton Institute, Invergowrie, Dundee, UK},\\
    and the Biopython Contributors\thanks{See \href{https://github.com/biopython/biopython/blob/master/CONTRIB.rst}{contributor listing on GitHub}.}}
\date{20\textsuperscript{th} Bioinformatics Open Source Conference (BOSC) 2019, Basel, Switzerland}

\begin{document}
\maketitle
\thispagestyle{empty}

\vspace{-0.2in}
\noindent
Website: \url{http://biopython.org} \\
Repository: \url{https://github.com/biopython/biopython} \\
License: Biopython License Agreement (BSD like, see \url{http://www.biopython.org/DIST/LICENSE}) \\

The Biopython Project is a long-running distributed collaborative effort,
supported by the Open Bioinformatics Foundation, which develops a freely
available Python library for biological computation \cite{AppNote}. This
talk will look ahead to the year to come, and give a summary of the project
news since the 1.72 release in June 2018, and the talk at GCCBOSC 2018.

While there were no major new modules introduced in Biopython 1.73
(December 2018) or Biopython 1.74 (expected May/June 2019), there have
been lots of incremental improvements.
In terms of lines of code changed, a substantial proportion have been
in-line documentation (Python docstrings), used to generate human readable
API documentation. While we are still using \texttt{epydoc} for this, our continuous
integration system has been generating more modern HTML output using \texttt{sphinx},
which we hope to host on our domain, or at Read The Docs, making this work
much more visible to the world. We have been using the tool \texttt{flake8}
with various plugins for this (as well as checking coding style), showing
a steady improvement in best practice compliance - every public API should
be documented this year.

In 2017 we started a re-licensing plan, to transition away
from our liberal but unique \emph{Biopython License Agreement} to the similar
but very widely used \emph{3-Clause BSD License}. We are reviewing the code
base authorship file-by-file, to gradually dual license the entire project.
All new contributions are dual licensed, and currently half the Python files
in the main library have been dual licensed.

Another important going effort is improving the unit test coverage. Sadly
This is currently fairly static at about $85\%$ (excluding online tests),
but can be viewed online at
\href{https://codecov.io/github/biopython/biopython/}{CodeCov.io}.

We are using GitHub-integrated continuous integration testing on Linux (using
\href{https://travis-ci.org/biopython/biopython/builds}{TravisCI}) and Windows
(using \href{https://ci.appveyor.com/project/biopython/biopython/history}{AppVeyor}),
including enforcing the Python PEP8 and PEP257 coding style guidelines.
We hope to be able to recommend a simple \texttt{git pre-commit} hook
for our contributors shortly, and have discussed the idea of adopting the
new yet popular Python code formatting style tool \texttt{black} to reduce
the human time costs in writing compliant code.

Looking further ahead, in 2020, in line with most major scientific Python
libraries, we will be dropping support for Python 2. See \url{https://python3statement.org/}

Finally, since our last update talk in June 2018, Biopython has had 32 named
contributors including 14 newcomers. This reflects our policy of trying to
encourage even small contributions. This brings our total named contributor
count to 248 since the project began, and looks likely to break 250 by our
20th Birthday in August 2019.

\begin{thebibliography}{}

\bibitem[Cock {\it et al}., 2009]{AppNote}Cock, P.J.A., Antao, T., Chang, J.T., Chapman, B.A., Cox, C.J., Dalke, A., Friedberg, I., Hamelryck, T., Kauff, F., Wilczynski, B., de Hoon, M.J. (2009) Biopython: freely available Python tools for computational molecular biology and bioinformatics. {\it Bioinformatics} {\bf 25}(11) 1422-3. \href{http://dx.doi.org/10.1093/bioinformatics/btp163}{doi:10.1093/bioinformatics/btp163}

\end{thebibliography}

\end{document}
